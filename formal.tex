% !TeX root = hott-online.tex

\titleformat{\chapter}[display]{\fontsize{23}{25}\fontseries{m}\fontshape{it}\selectfont}{\chaptertitlename}{20pt}{\fontsize{35}{35}\fontseries{b}\fontshape{n}\selectfont}
\chapter{Formal type theory}
\label{cha:rules}

\index{formal!type theory|(}%
\index{type theory!formal|(}%
\index{rules of type theory|(}%

Just as one can develop mathematics in set theory without explicitly using the axioms of Zermelo--Fraenkel set theory, 
in this book we have developed mathematics in univalent foundations without explicitly referring to a formal
system of homotopy type theory. Nevertheless, it is important to \emph{have} a
precise description of homotopy type theory as a formal system in order to, for example,
%
\begin{itemize}
\item state and prove its metatheoretic properties, including logical
consistency,
\item construct models, e.g.\  in simplicial sets, model categories, higher toposes,
etc., and
\item implement it in proof assistants like \Coq or \Agda.
  \index{proof!assistant}
\end{itemize}
%
Even the logical consistency\index{consistency} of homotopy type theory, namely that in the empty context there is no term $a:\emptyt$, is not obvious: if we had erroneously
chosen a definition of equivalence for which $\eqv{\emptyt}{\unit}$, then
univalence would imply that $\emptyt$ has an element, since $\unit$ does.
Nor is it obvious that, for example, our definition of $\Sn^1$ as a higher
inductive type yields a type which behaves like the ordinary circle.

There are two aspects of type theory which we must pin down before addressing
such questions. Recall from the Introduction that type theory
comprises a set of rules specifying when the judgments $a:A$ and $a\jdeq a':A$
hold---for example, products are characterized by the rule that whenever $a:A$
and $b:B$, $(a,b):A\times B$. To make this precise, we must first define
precisely the syntax of terms---the objects $a,a',A,\dots$ which these judgments
relate; then, we must define precisely the judgments and their rules of
inference---the manner in which judgments can be derived from other judgments.

In this appendix, we present two formulations of Martin-L\"{o}f type
theory, and of the extensions that constitute homotopy type theory.
The first presentation (\cref{sec:syntax-informally}) describes the syntax of
terms and the forms of judgments as an extension of the untyped
$\lambda$-calculus, while leaving the rules of inference informal.
The second (\cref{sec:syntax-more-formally}) defines the terms, judgments,
and rules of inference inductively in the style of natural deduction, as
is customary in much type-theoretic literature.

\section*{Preliminaries}
\label{sec:formal-prelim}


In \cref{cha:typetheory}, we presented the two basic \define{judgments}
\index{judgment}
of type theory. The first, $a:A$, asserts that a term $a$ has type $A$. The second,
$a\jdeq b:A$, states that the two terms $a$ and $b$ are \define{judgmentally
equal}%
\index{equality!judgmental}
\index{judgmental equality}
at type $A$. These judgments are inductively defined by a set of
inference rules described in \cref{sec:syntax-more-formally}.

To construct an element $a$ of a type $A$ is to derive $a:A$; in the book, we
give informal arguments which describe the construction of $a$, but formally,
one must specify a precise term $a$ and a full derivation that $a:A$.

However, the main difference between the presentation of type theory in the book
and in this appendix is that here judgments are explicitly
formulated in an ambient \define{context},
\index{context}
or list of assumptions, of the form
\[
  x_1:A_1, x_2:A_2,\dots,x_n:A_n.
\]
An element $x_i : A_i$ of the context expresses the assumption that the
variable
\index{variable}%
$x_i$ has type $A_i$. The variables $x_1, \ldots, x_n$ appearing in
the context must be distinct. We abbreviate contexts with the letters $\Gamma$
and $\Delta$.

The judgment $a:A$ in context $\Gamma$ is written 
\[ \oftp\Gamma aA \]
and means that $a:A$ under the assumptions listed in $\Gamma$. When the list of
assumptions is empty, we write simply
\[ \oftp{}aA \]
or
\[ \oftp\emptyctx aA \]
where $\emptyctx$ denotes the empty context. The same applies to the equality
judgment
\[
  \jdeqtp\Gamma{a}{b}{A}
\]

However, such judgments are sensible only for \define{well-formed} contexts,
\index{context!well-formed}%
a notion captured by our third and final judgment
\[
  \wfctx{(x_1:A_1, x_2:A_2,\dots,x_n:A_n)}
\]
expressing that each $A_i$ is a type in the context $x_1:A_1,
x_2:A_2,\dots,x_{i-1}:A_{i-1}$.  In particular, therefore, if $\oftp\Gamma aA$ and
$\wfctx\Gamma$, then we know that each $A_i$ contains only the variables
$x_1,\dots,x_{i-1}$, and that $a$ and $A$ contain only the variables
$x_1,\dots,x_n$.
\index{variable!in context}

In informal mathematical presentations, the context is
implicit. At each point in a proof, the mathematician knows which
variables are available and what types they have, either by historical
convention ($n$ is usually a number, $f$ is a function, etc.) or
because variables are explicitly introduced with sentences such as
``let $x$ be a real number''. We discuss some benefits of using explicit
contexts in \cref{sec:more-formal-pi,sec:more-formal-sigma}.

We write $B[a/x]$ for the \define{substitution}
\index{substitution}%
of a term $a$ for free occurrences of
the variable~$x$ in the term $B$, with possible capture-avoiding
renaming of bound variables,
\index{variable!and substitution}%
as discussed in
\cref{sec:function-types}. The general form of substitution
%
\[
   B[a_1,\dots,a_n/x_1,\dots,x_n]
\]
%
substitutes expressions $a_1,\dots,a_n$ for the variables
$x_1,\dots,x_n$ simultaneously.

To \define{bind a variable $x$ in an expression $B$}
\indexdef{variable!bound}%
means to incorporate both of them into a larger expression, called an \define{abstraction},
\indexdef{abstraction}%
whose purpose is to express the fact that $x$ is ``local'' to $B$, i.e., it
is not to be confused with other occurrences of $x$ appearing
elsewhere. Bound variables are familiar to programmers, but less so to mathematicians.
Various notations are used for binding, such as $x \mapsto B$,
$\lam x B$, and $x \,.\, B$, depending on the situation. We may write $C[a]$ for the
substitution of a term $a$ for the variable in the abstracted expression, i.e.,
we may define $(x.B)[a]$ to be $B[a/x]$. As discussed in
\cref{sec:function-types}, changing the name of a bound variable everywhere within an expression (``$\alpha$-conversion'')
\index{alpha-conversion@$\alpha $-conversion}%
does not change the expression. Thus, to be very
precise, an expression is an equivalence class of syntactic forms
which differ in names of bound variables.

One may also regard each variable $x_i$ of a judgment
\[
  x_1:A_1, x_2:A_2,\dots,x_n:A_n \vdash a : A
\]
to be bound in its \define{scope},
\indexdef{variable!scope of}%
\index{scope}%
consisting of the expressions $A_{i+1},
\ldots, A_n$, $a$, and $A$.

\section{The first presentation}
\label{sec:syntax-informally}

The objects and types of our type theory may be written as terms using
the following syntax, which is an extension of $\lambda$-calculus with
\emph{variables} $x, x',\dots$,
\index{variable}%
\emph{primitive constants}
\index{primitive!constant}%
\index{constant!primitive}%
$c,c',\dots$, \emph{defined constants}\index{constant!defined} $f,f',\dots$, and term forming
operations
%
\[
  t \production x \mid \lam{x} t \mid t(t') \mid c \mid f
\]
%
The notation used here means that a term $t$ is either a variable $x$, or it
has the form $\lam{x} t$ where $x$ is a variable and $t$ is a term, or it has
the form $t(t')$ where $t$ and $t'$ are terms, or it is a primitive constant
$c$, or it is a defined constant $f$. The syntactic markers '$\lambda$', '(',
')', and '.' are punctuation for guiding the human eye.

We use $t(t_1,\dots,t_n)$ as an abbreviation for the repeated application
$t(t_1)(t_2)\dots (t_n)$. We may also use \emph{infix}\index{infix notation} notation, writing $t_1\;
\star\; t_2$ for $\star(t_1,t_2)$ when $\star$ is a primitive or defined
constant.

Each defined constant has zero, one or more \define{defining equations}.
\index{equation, defining}%
\index{defining equation}%
There are two kinds of defined constant. An \emph{explicit}
\index{constant!explicit}
defined constant $f$ has a single defining equation
  \[ f(x_1,\dots,x_n)\defeq t,\]
where $t$ does not involve $f$. 
%
For example, we might introduce the explicit defined constant $\circ$ with defining equation
  \[ \circ (x,y)(z) \defeq x(y(z)),\]
and use infix notation $x\circ y$ for $\circ(x,y)$. This of course is just composition of functions.

The second kind of defined constant is used to specify a (parameterized) mapping
$f(x_1,\dots,x_n,x)$, where $x$ ranges over a type whose elements are generated
by zero or more primitive constants.  For each such primitive constant $c$ there
is a defining equation of the form
\[
  f(x_1,\dots,x_n,c(y_1,\dots,y_m)) \defeq t,
\]
where $f$ may occur in $t$, but only in such a way that it is clear that the
equations determine a totally defined function. The paradigm examples of such
defined functions are the functions defined by primitive recursion on the
natural numbers. We may call this kind of definition of a function a \emph{total
  recursive definition}.
\index{total!recursive definition}%
In computer science and logic this kind of definition
of a function on a recursive data type has been called a \define{definition by
  structural recursion}.
\index{definition!by structural recursion}%
\index{structural!recursion}%
\index{recursion!structural}%

\define{Convertibility}
\index{convertibility of terms}%
\index{term!convertibility of}%
$t \conv t'$ between terms $t$
and $t'$ is the equivalence relation generated by the defining equations for constants,
the computation rule\index{computation rule!for function types}
%
\[
  (\lam{x} t)(u) \defeq t[u/x],
\]
%
and the rules which make it a \emph{congruence} with respect to application and $\lambda$-abstraction\index{lambda abstraction@$\lambda$-abstraction}:
%
\begin{itemize}
\item if $t \conv t'$ and $s \conv s'$ then $t(s) \conv t'(s')$, and
\item if $t \conv t'$ then $(\lam{x} t) \conv (\lam{x} t')$.
\end{itemize}
\noindent
The equality judgment $t \jdeq u : A$ is then derived by the following single rule:
%
\begin{itemize}
\item if $t:A$, $u:A$, and $t \conv u$, then $t \jdeq u : A$.
\end{itemize}
%
Judgmental equality is an equivalence relation.

Note that the type theory of this presentation diverges from that used in the main body of the text in not including the judgmental uniqueness principle $f \jdeq (\lam{x} f(x))$ for functions.
Such an equality requires that judgmental equality be sensitive to the type of the terms involved, as this equality only makes sense when $f$ is known to be a function, whereas in this presentation the convertibility relation is type-independent.
The second presentation in \cref{sec:syntax-more-formally} includes the uniqueness principle.


\subsection{Type universes}

We postulate a hierarchy of \define{universes} denoted by primitive constants
\index{type!universe}
%
\begin{equation*}
  \UU_0, \quad \UU_1, \quad  \UU_2, \quad \ldots
\end{equation*}
%
The first two rules for universes say that they form a cumulative hierarchy of types:
%
\begin{itemize}
\item $\UU_m : \UU_n$ for $m < n$,
\item if $A:\UU_m$ and $m \le n$, then $A:\UU_n$,
\end{itemize}
%
and the third expresses the idea that an object of a universe can serve as a type and stand to the
right of a colon in judgments:
%
\begin{itemize}
\item if $\Gamma \vdash A : \UU_n$, and $x$ is a new variable,%
\footnote{By ``new'' we mean that it does not appear in $\Gamma$ or $A$.}
then $\vdash (\Gamma, x:A)\; \ctx$.
\end{itemize}
%
In the body of the book, an equality judgment $A \jdeq B : \UU_n$ between types
$A$ and $B$ is usually abbreviated to $A \jdeq B$. This is an instance of
typical ambiguity\index{typical ambiguity}, as we can always switch to a larger universe, which however does not affect the validity of the judgment.

The following conversion rule allows us to replace a type by one equal to it in a typing judgment:
%
\begin{itemize}
\item if $a:A$ and $A \jdeq B$ then $a:B$.
\end{itemize}

\subsection{Dependent function types (\texorpdfstring{$\Pi$}{Π}-types)}

We introduce a primitive constant $c_\Pi$, but write
$c_\Pi(A,\lam{x} B)$ as $\tprd{x:A}B$. Judgments concerning
such expressions and expressions of the form $\lam{x} b$ are introduced by the following rules:
%
\begin{itemize}
\item if $\Gamma \vdash A:\UU_n$ and $\Gamma,x:A \vdash B:\UU_n$, then $\Gamma \vdash \tprd{x:A}B : \UU_n$
\item if $\Gamma, x:A \vdash b:B$ then $\Gamma \vdash (\lam{x} b) : (\tprd{x:A} B)$
\item if $\Gamma\vdash g:\tprd{x:A} B$ and $\Gamma\vdash t:A$ then $\Gamma\vdash g(t):B[t/x]$
\end{itemize}
%
If $x$ does not occur freely in $B$, we abbreviate $\tprd{x:A} B$ as the non-dependent function type 
$A\rightarrow B$ and derive the following rule:
%
\begin{itemize}
\item if $\Gamma\vdash g:A \rightarrow B$ and $\Gamma\vdash t:A$ then $\Gamma\vdash g(t):B$
\end{itemize}
Using non-dependent function types and leaving implicit the context $\Gamma$, the rules above can be written in the following alternative style that we use in the rest of this section of the appendix:
%
\begin{itemize}
\item if $A:\UU_n$ and $B:A\to\UU_n$, then $\tprd{x:A}B(x) : \UU_n$
\item if $x:A \vdash b:B$ then $ \lam{x} b : \tprd{x:A} B(x)$
\item if $g:\tprd{x:A} B(x)$ and $t:A$ then $g(t):B(t)$
\end{itemize}
%

\subsection{Dependent pair types (\texorpdfstring{$\Sigma$}{Σ}-types)}

We introduce primitive constants $c_\Sigma$ and $c_{\mathsf{pair}}$. An
expression of the form $c_\Sigma(A,\lam{a} B)$ is written as $\sm{a:A}B$,
and an expression of the form $c_{\mathsf{pair}}(a,b)$ is written as $\tup
a b$. We write $A\times B$ instead of $\sm{x:A} B$ if $x$ is not free in $B$.

Judgments concerning such expressions are introduced by the following
rules:
%
\begin{itemize}
\item if $A:\UU_n$ and $B: A \rightarrow \UU_n$, then $\sm{x:A}B(x) : \UU_n$
\item if, in addition, $a:A$ and $b:B(a)$, then $\tup a b:\sm{x:A}B(x)$
\end{itemize}
%
If we have $A$ and $B$ as above, $C : (\sm{x:A}B(x)) \rightarrow \UU_m$, and
\[
  d:\tprd{x:A}{y:B(x)} C(\tup x y)
\]
we can introduce a defined constant 
\[
  f:\tprd{p:\sm{x:A}B(x)} C(p)
\]
with the defining equation
\[
  f(\tup x y)\defeq d(x,y).
\]
%
Note that $C$, $d$, $x$, and $y$ may contain extra implicit parameters $x_1,\ldots,x_n$ if they were obtained in some non-empty context; therefore, the fully explicit recursion schema is
%
\begin{narrowmultline*}
 f(x_1,\dots,x_n,\tup{x(x_1,\dots,x_n)}{y(x_1,\dots,x_n)}) \defeq
 \narrowbreak
 d(x_1,\dots,x_n,\tup{x(x_1,\dots,x_n)}{y(x_1,\dots,x_n)}).
\end{narrowmultline*}

\subsection{Coproduct types}

We introduce primitive constants $c_+$, $c_\inlsym$, and $c_\inrsym$.
We write $A+B$ instead of $c_+(A,B)$, $\inl(a)$ instead of
$c_\inlsym(a)$, and $\inr(a)$ instead of $c_\inrsym(a)$:
%
\begin{itemize}
\item if $A,B : \UU_n$ then $A + B : \UU_n$
\item moreover, $\inl: A \rightarrow A+B$ and $\inr: B \rightarrow A+B$
\end{itemize}
%
If we have $A$ and $B$ as above, $C : A+B \rightarrow \UU_m$, 
$d:\tprd{x:A} C(\inl(x))$, and $e:\tprd{y:B} C(\inr(y))$,
then we can introduce a defined constant $f:\tprd{z:A+B}C(z)$ with the defining equations
%
\begin{equation*}
  f(\inl(x)) \defeq d(x)
  \qquad\text{and}\qquad
  f(\inr(y)) \defeq e(y).
\end{equation*}

\subsection{The finite types}

We introduce primitive constants $\ttt$, $\emptyt$, $\unit$, satisfying the following rules:
%
\begin{itemize}
\item $\emptyt : \UU_0$, $\unit : \UU_0$
\item $\ttt:\unit$
\end{itemize}

Given $C : \emptyt \rightarrow \UU_n$ we can introduce a defined constant $f:\tprd{x:\emptyt} C(x)$, with no defining equations.

Given $C : \unit \rightarrow \UU_n$ and $d : C(\ttt)$ we can introduce a defined constant $f:\tprd{x:\unit} C(x)$, with defining equation $f(\ttt) \defeq d$.

\subsection{Natural numbers}

The type of natural numbers is obtained by introducing primitive constants
$\N$, $0$, and $\suc$ with the following rules:
%
\begin{itemize}
  \item $\N : \UU_0$,
  \item $0:\N$,
  \item $\suc:\N\rightarrow \N$.
\end{itemize}
%
Furthermore, we can define functions by primitive recursion. If we have
$C : \N \rightarrow \UU_k $ we can introduce a defined constant $f:\tprd{x:\N}C(x)$ whenever we have
%
\begin{align*}
  d & : C(0) \\
  e & : \tprd{x:\N}(C(x)\rightarrow C(\suc (x)))
\end{align*}
%
with the defining equations
%
\begin{equation*}
  f(0) \defeq d
  \qquad\text{and}\qquad
  f(\suc (x)) \defeq e(x,f(x)).
\end{equation*}

\subsection{\texorpdfstring{$W$}{W}-types}

For $W$-types we introduce primitive constants $c_\wtypesym$ and $c_\suppsym$.
An expression of the form $c_\wtypesym(A,\lam{x} B)$ is written as
$\wtype{x:A}B$, and an expression of the form $c_\suppsym(x,u)$ is written
as $\supp(x,u)$:
%
\begin{itemize}
\item if $A:\UU_n$ and $B: A \rightarrow \UU_n$, then $\wtype{x:A}B(x) : \UU_n$
\item if moreover, $a:A$ and $u:B(a)\rightarrow \wtype{x:A}B(x)$ then $\supp(a,u):\wtype{x:A}B(x)$.
\end{itemize}
% 
Here also we can define functions by total recursion. If we have $A$ and $B$
as above and $C : (\wtype{x:A}B(x)) \rightarrow \UU_m$, then we can introduce a defined constant
$f:\tprd{z:\wtype{x:A}B(x)} C(z)$ whenever we have
\[
  d:\tprd{a:A}{u:B(a) \rightarrow \wtype{x:A}B(x)}((\tprd{y:B(a)}C(u(y))) \rightarrow C(\supp(a,u)))
\]
with the defining equation
\[
  f(\supp(a,u)) \defeq d(a,u,f\circ u).
\]

\subsection{Identity types}

We introduce primitive constants $c_\idsym$ and $c_{\refl{}}$. We write
$\id[A] a b$ for $c_\idsym(A,a,b)$ and $\refl a$ for $c_{\refl{}}(A,a)$, when
$a:A$ is understood:
%
\begin{itemize}
\item If $A : \UU_n$, $a:A$, and $b:A$ then $\id[A] a b : \UU_n$.
\item If $a:A$ then $\refl a :\id[A] a a $.
\end{itemize}
%
Given $a:A$, if $y:A, z:\id[A] a y \vdash C : \UU_m$ and 
$\vdash d:C[a,\refl{a}/y,z]$ then we can introduce a defined constant 
\[
  f:\tprd{y:A}{z:\id[A] a y} C
\]
with defining equation
\[
  f(a,\refl{a})\defeq d.
\]

\section{The second presentation}
\label{sec:syntax-more-formally}

In this section, there are three kinds of judgments 
\begin{mathpar}
\wfctx\Gamma
\and
\oftp\Gamma{a}{A}
\and
\jdeqtp\Gamma{a}{a'}{A}
\end{mathpar}
which we specify by providing inference rules for deriving them. A typical \define{inference rule}
\indexsee{inference rule}{rule}%
\indexdef{rule}%
has the form
%
\begin{equation*}
  \inferrule*[right=\textsc{Name}]
  {\mathcal{J}_1 \\ \cdots \\ \mathcal{J}_k}
  {\mathcal{J}}
\end{equation*}
%
It says that we may derive the \define{conclusion} $\mathcal{J}$, provided that we have
already derived the \define{hypotheses} $\mathcal{J}_1, \ldots, \mathcal{J}_k$.
(Note that, being judgments rather than types, these are not hypotheses \emph{internal} to the type theory in the sense of \cref{sec:types-vs-sets}; they are instead hypotheses in the deductive system, i.e.\ the metatheory.)
On the
right we write the \textsc{Name} of the rule, and there may be extra side conditions that
need to be checked before the rule is applicable.

A \define{derivation}
\index{derivation}%
of a judgment is a tree constructed from such inference
rules, with the judgment at the root of the tree. For example, with the rules given below, the following is a derivation of
$\oftp{\emptyctx}{\lamu{x:\unit} x}{\unit\to\unit}$.
%
\begin{mathpar}
\inferrule*[right=$\Pi$-\rintro]
  {\inferrule*[right=$\Vble$]
    {\inferrule*[right=\ctx-\textsc{ext}]
      {\inferrule*[right=$\unit$-\rform]
        {\inferrule*[right=\ctx-\textsc{emp}]
          {\ }
          {\wfctx {\emptyctx}}}
        {\oftp{}{\unit}{\UU_0}}}
      {\wfctx {\tmtp x\unit}}}
   {\oftp{\tmtp x\unit}{x}{\unit}}}
 {\oftp{\emptyctx}{\lamu{x:\unit} x}{\unit\to\unit}}
\end{mathpar}

\subsection{Contexts}
\label{subsec:contexts}

\index{context}%
A context is a list
%
\begin{equation*}
  \tmtp{x_1}{A_1}, \tmtp{x_2}{A_2}, \ldots, \tmtp{x_n}{A_n}
\end{equation*}
%
which indicates that the distinct variables
\index{variable}%
$x_1, \ldots, x_n$ are assumed to have types $A_1, \ldots, A_n$, respectively. The list may be empty. We abbreviate contexts with the letters $\Gamma$ and $\Delta$, and we may juxtapose them to form larger contexts.

The judgment $\wfctx{\Gamma}$ formally expresses the fact that $\Gamma$ is a well-formed context, and is governed by the rules of inference
%
\begin{mathpar}
  \inferrule*[right=\ctx-\textsc{emp}]
  {\ }
  {\wfctx\emptyctx}
\and
  \inferrule*[right=\ctx-\textsc{ext}]
  {\oftp{\tmtp{x_1}{A_1}, \ldots, \tmtp{x_{n-1}}{A_{n-1}}}{A_n}{\UU_i}}
  {\wfctx{(\tmtp{x_1}{A_1}, \ldots, \tmtp{x_n}{A_n})}}
\end{mathpar}
%
with a side condition for the second rule: the variable $x_n$ must be distinct from the variables $x_1, \ldots, x_{n-1}$.
Note that the hypothesis and conclusion of $\ctx$-\textsc{ext} are judgments of different forms: the hypothesis says that in the context of variables $x_1, \ldots, x_{n-1}$, the expression $A_n$ has type $\UU_i$; while the conclusion says that the extended context $(\tmtp{x_1}{A_1}, \ldots, \tmtp{x_n}{A_n})$ is well-formed.
(It is a meta-theoretic property of the system that if $\oftp{\tmtp{x_1}{A_1}, \ldots, \tmtp{x_{n}}{A_{n}}}{b}{B}$ is derivable, then the context $(\tmtp{x_1}{A_1}, \ldots, \tmtp{x_{n}}{A_{n}})$ must be well-formed; thus $\ctx$-\textsc{ext} does not need to hypothesize well-formedness of the context to the left of $x_n$.)

\subsection{Structural rules}

\index{structural!rules|(}%
\index{rule!structural|(}%

The fact that the context holds assumptions is expressed by the rule which says that we may derive those typing judgments which are listed in the context:
%
\begin{mathpar}
  \inferrule*[right=$\Vble$]
  {\wfctx {(\tmtp{x_1}{A_1}, \ldots, \tmtp{x_n}{A_n})} }
  {\oftp{\tmtp{x_1}{A_1}, \ldots, \tmtp{x_n}{A_n}}{x_i}{A_i}}
\end{mathpar}
%
As with $\ctx$-\textsc{ext}, the hypothesis and conclusion of the rule $\Vble$ are judgments of different forms, only now they are reversed: we start with a well-formed context and derive a typing judgment.

The following important principles, called \define{substitution}
\indexdef{rule!of substitution}%
and
\define{weakening},
\indexdef{rule!of weakening}%
need not be explicitly assumed. Rather, it is possible to
show, by induction on the structure of all possible derivations, that whenever
the hypotheses of these rules are derivable, their conclusion is also
derivable.\footnote{Such rules are called \define{admissible}\indexdef{rule!admissible}\indexsee{admissible!rule}{rule, admissible}.}
For the typing judgments these principles are manifested as
%
\begin{mathpar}
  \inferrule*[right=$\Subst_1$]
  {\oftp\Gamma{a}{A} \\ \oftp{\Gamma,\tmtp xA,\Delta}{b}{B}}
  {\oftp{\Gamma,\Delta[a/x]}{b[a/x]}{B[a/x]}}
\and
  \inferrule*[right=$\Weak_1$]
  {\oftp\Gamma{A}{\UU_i} \\ \oftp{\Gamma,\Delta}{b}{B}}
  {\oftp{\Gamma,\tmtp xA,\Delta}{b}{B}}
\end{mathpar}
and for judgmental equalities they become
\begin{mathpar}
  \inferrule*[right=$\Subst_2$]
  {\oftp\Gamma{a}{A} \\ \jdeqtp{\Gamma,\tmtp xA,\Delta}{b}{c}{B}}
  {\jdeqtp{\Gamma,\Delta[a/x]}{b[a/x]}{c[a/x]}{B[a/x]}}
\and
  \inferrule*[right=$\Weak_2$]
  {\oftp\Gamma{A}{\UU_i} \\ \jdeqtp{\Gamma,\Delta}{b}{c}{B}}
  {\jdeqtp{\Gamma,\tmtp xA,\Delta}{b}{c}{B}}
\end{mathpar}
%
In addition to the judgmental equality rules given for each type former, we also
assume that judgmental equality is an equivalence relation respected by typing.
\begin{mathparpagebreakable}
  \inferrule*{\oftp\Gamma{a}{A}}{\jdeqtp\Gamma{a}{a}{A}}
\and
  \inferrule*{\jdeqtp\Gamma{a}{b}{A}}{\jdeqtp\Gamma{b}{a}{A}}
\and
  \inferrule*{\jdeqtp\Gamma{a}{b}{A} \\ \jdeqtp\Gamma{b}{c}{A}}{\jdeqtp\Gamma{a}{c}{A}}
\and
  \inferrule*{\oftp\Gamma{a}{A} \\ \jdeqtp\Gamma{A}{B}{\UU_i}}{\oftp\Gamma{a}{B}}
\and
  \inferrule*{\jdeqtp\Gamma{a}{b}{A} \\ \jdeqtp\Gamma{A}{B}{\UU_i}}{\jdeqtp\Gamma{a}{b}{B}}
\end{mathparpagebreakable}
%
Additionally, for all the type formers below, we assume rules stating that each constructor preserves definitional equality in each of its arguments; for instance, along with the $\Pi$-\rintro\ rule, we assume the rule
\[
  \inferrule*[right=$\Pi$-\rintro-eq]
  {\oftp\Gamma{A}{\UU_i} \\
   \oftp{\Gamma,\tmtp xA}{B}{\UU_i} \\
   \jdeqtp{\Gamma,\tmtp xA}{b}{b'}{B}}
  {\jdeqtp\Gamma{\lamu{x:A} b}{\lamu{x:A'} b'}{\tprd{x:A} B}}
\]
However, we omit these rules for brevity.

\index{rule!structural|)}%
\index{structural!rules|)}%

\subsection{Type universes}

\index{type!universe}%

We postulate an infinite hierarchy of type universes
%
\begin{equation*}
  \UU_0, \quad \UU_1, \quad  \UU_2, \quad \ldots
\end{equation*}
%
Each universe is contained in the next, and any type in $\UU_i$ is also in $\UU_{i+1}$:
%
\begin{mathpar}
\inferrule*[right=\UU-\textsc{intro}]
  {\wfctx \Gamma }
  {\oftp\Gamma{\UU_i}{\UU_{i+1}}}
\and
\inferrule*[right=\UU-\textsc{cumul}]
  {\oftp\Gamma{A}{\UU_i}}
  {\oftp\Gamma{A}{\UU_{i+1}}}
\end{mathpar}
%
We shall set up the rules of type theory in such a way that $\oftp\Gamma{a}{A}$
implies $\oftp\Gamma{A}{\UU_i}$ for some $i$. In other words, if $A$ plays the role of a type then it is in some universe. Another property of our type system is that $\jdeqtp\Gamma{a}{b}{A}$
implies $\oftp\Gamma{a}{A}$ and $\oftp\Gamma{b}{A}$.

\subsection{Dependent function types (\texorpdfstring{$\Pi$}{Π}-types)}
\label{sec:more-formal-pi}

\index{type!dependent function}%
\index{type!function}%

In \cref{sec:function-types}, we introduced non-dependent functions $A\to B$ in
order to define a family of types as a function $\lam{x:A} B:A\to\UU_i$, which
then gives rise to a type of dependent functions $\tprd{x:A} B$. But with explicit contexts
we may replace $\lam{x:A} B:A\to\UU_i$ with the judgment
%
\begin{equation*}
  \oftp{\tmtp xA}{B}{\UU_i}.
\end{equation*}
%
Consequently, we may define dependent functions directly, without reference to non-dependent ones. This way we follow the general principle that each type former, with its constants and rules, should be introduced independently of all other type formers.
%
In fact, henceforth each type former is introduced systematically by:
\begin{itemize}
\item a \define{formation rule}, stating when the type former can be applied;\index{formation rule}\index{rule!formation}
\item some \define{introduction rules}, stating how to inhabit the type;\index{introduction rule}\index{rule!introduction}
\item \define{elimination rules}, or an induction principle, stating how to use an
  element of the type;
  \index{induction principle}\index{eliminator}
\item \define{computation rules}, which are judgmental equalities explaining what happens when elimination rules are applied to results of introduction rules;
  \index{computation rule}
  \indexsee{rule!computation}{computation rule}
\item optional \define{uniqueness principles}, which are judgmental equalities explaining how every element of the type is uniquely determined by the results of elimination rules applied to it.
  \index{uniqueness!principle}
  \indexsee{principle!uniqueness}{uniqueness principle}
\end{itemize}
(See also \cref{rmk:introducing-new-concepts}.)

For the dependent function type these rules are:
%
\begin{mathparpagebreakable}
  \def\premise{\oftp{\Gamma}{A}{\UU_i} \and \oftp{\Gamma,\tmtp xA}{B}{\UU_i}}
  \inferrule*[right=$\Pi$-\rform]
    \premise
    {\oftp\Gamma{\tprd{x:A}B}{\UU_i}}
\and
  \inferrule*[right=$\Pi$-\rintro]
  {\oftp{\Gamma,\tmtp xA}{b}{B}}
  {\oftp\Gamma{\lam{x:A} b}{\tprd{x:A} B}}
\and
  \inferrule*[right=$\Pi$-\relim]
  {\oftp\Gamma{f}{\tprd{x:A} B} \\ \oftp\Gamma{a}{A}}
  {\oftp\Gamma{f(a)}{B[a/x]}}
\and
  \inferrule*[right=$\Pi$-\rcomp]
  {\oftp{\Gamma,\tmtp xA}{b}{B} \\ \oftp\Gamma{a}{A}}
  {\jdeqtp\Gamma{(\lam{x:A} b)(a)}{b[a/x]}{B[a/x]}}
\and
  \inferrule*[right=$\Pi$-\runiq]
  {\oftp\Gamma{f}{\tprd{x:A} B}}
  {\jdeqtp\Gamma{f}{(\lamu{x:A}f(x))}{\tprd{x:A} B}}
\end{mathparpagebreakable}

The expression $\lam{x:A} b$ binds free occurrences of $x$ in $b$, as does $\tprd{x:A} B$ for
$B$.

When $x$ does not occur freely in $B$ so that $B$ does not depend on $A$, we obtain as a
special case the ordinary function type $A\to B \defeq \tprd{x:A} B$. We take this as the \emph{definition} of $\to$.

We may abbreviate an expression $\lam{x:A} b$ as $\lamu{x:A} b$, with the understanding
that the omitted type $A$ should be filled in appropriately before type-checking.

\subsection{Dependent pair types (\texorpdfstring{$\Sigma$}{Σ}-types)}
\label{sec:more-formal-sigma}

\index{type!dependent pair}%
\index{type!product}%

In \cref{sec:sigma-types}, we needed $\to$ and $\prdsym$ types in order to
define the introduction and elimination rules for $\smsym$; as with $\prdsym$, contexts allow us to state the rules for $\smsym$ independently.
Recall that the elimination rule for a positive type such as $\Sigma$ is called \emph{induction} and denoted by $\ind{}$.
%
\begin{mathparpagebreakable}
  \def\premise{\oftp{\Gamma}{A}{\UU_i} \and \oftp{\Gamma,\tmtp xA}{B}{\UU_i}}
  \inferrule*[right=$\Sigma$-\rform]
    \premise
    {\oftp\Gamma{\tsm{x:A} B}{\UU_i}}
  \and
  \inferrule*[right=$\Sigma$-\rintro]
    {\oftp{\Gamma, \tmtp x A}{B}{\UU_i} \\
     \oftp\Gamma{a}{A} \\ \oftp\Gamma{b}{B[a/x]}}
    {\oftp\Gamma{\tup ab}{\tsm{x:A} B}}
  \and
  \inferrule*[right=$\Sigma$-\relim]
    {\oftp{\Gamma, \tmtp z {\tsm{x:A} B}}{C}{\UU_i} \\
     \oftp{\Gamma,\tmtp x A,\tmtp y B}{g}{C[\tup x y/z]} \\
     \oftp\Gamma{p}{\tsm{x:A} B}}
    {\oftp\Gamma{\ind{\tsm{x:A} B}(z.C,x.y.g,p)}{C[p/z]}}
  \and
  \inferrule*[right=$\Sigma$-\rcomp]
    {\oftp{\Gamma, \tmtp z {\tsm{x:A} B}}{C}{\UU_i} \\
     \oftp{\Gamma, \tmtp x A, \tmtp y B}{g}{C[\tup x y/z]} \\\\
     \oftp\Gamma{a}{A} \\ \oftp\Gamma{b}{B[a/x]}}
    {\jdeqtp\Gamma{\ind{\tsm{x:A} B}(z.C,x.y.g,\tup{a}{b})}{g[a,b/x,y]}{C[\tup {a} {b}/z]}}
\end{mathparpagebreakable}
%
The expression $\tsm{x:A} B$ binds free occurrences of $x$ in $B$. Furthermore, because
$\ind{\tsm{x:A} B}$ has some arguments with free variables beyond those in $\Gamma$,
we bind (following the variable names above) $z$ in $C$, and $x$ and $y$ in $g$.
These bindings are written as $z.C$ and $x.y.g$, to indicate the names of the bound
variables.
\index{variable!bound}%
In particular, we treat $\ind{\tsm{x:A} B}$ as a primitive,
two of whose arguments contain binders; this is superficially similar to, but
different from, $\ind{\tsm{x:A} B}$ being a function that takes functions as
arguments.

When $B$ does not contain free occurrences of $x$, we obtain as a special case
the cartesian product $A \times B \defeq \tsm{x:A} B$. We take this
as the \emph{definition} of the cartesian product.

Notice that we don't postulate a judgmental uniqueness principle for $\Sigma$-types, even
though we could have; see \cref{thm:eta-sigma} for a proof of the corresponding
propositional uniqueness principle.

\subsection{Coproduct types}

\index{type!coproduct}%

\begin{mathparpagebreakable}
  \inferrule*[right=$+$-\rform]
  {\oftp\Gamma{A}{\UU_i} \\ \oftp\Gamma{B}{\UU_i}}
  {\oftp\Gamma{A+B}{\UU_i}}
\\
  \inferrule*[right=$+$-\rintro${}_1$]
  {\oftp\Gamma{A}{\UU_i} \\ \oftp\Gamma{B}{\UU_i} \\\\ \oftp\Gamma{a}{A}}
  {\oftp\Gamma{\inl(a)}{A+B}}
\and
  \inferrule*[right=$+$-\rintro${}_2$]
  {\oftp\Gamma{A}{\UU_i} \\ \oftp\Gamma{B}{\UU_i} \\\\ \oftp\Gamma{b}{B}}
  {\oftp\Gamma{\inr(b)}{A+B}}
\\
  \inferrule*[right=$+$-\relim]
  {\oftp{\Gamma,\tmtp z{(A+B)}}{C}{\UU_i} \\\\
   \oftp{\Gamma,\tmtp xA}{c}{C[\inl(x)/z]} \\
   \oftp{\Gamma,\tmtp yB}{d}{C[\inr(y)/z]} \\\\
   \oftp\Gamma{e}{A+B}}
  {\oftp\Gamma{\ind{A+B}(z.C,x.c,y.d,e)}{C[e/z]}}
\and
  \inferrule*[right=$+$-\rcomp${}_1$]
  {\oftp{\Gamma,\tmtp z{(A+B)}}{C}{\UU_i} \\
   \oftp{\Gamma,\tmtp xA}{c}{C[\inl(x)/z]} \\
   \oftp{\Gamma,\tmtp yB}{d}{C[\inr(y)/z]} \\\\
   \oftp\Gamma{a}{A}}
  {\jdeqtp\Gamma{\ind{A+B}(z.C,x.c,y.d,\inl(a))}{c[a/x]}{C[\inl(a)/z]}}
\and
  \inferrule*[right=$+$-\rcomp${}_2$]
  {\oftp{\Gamma,\tmtp z{(A+B)}}{C}{\UU_i} \\
   \oftp{\Gamma,\tmtp xA}{c}{C[\inl(x)/z]} \\
   \oftp{\Gamma,\tmtp yB}{d}{C[\inr(y)/z]} \\\\
   \oftp\Gamma{b}{B}}
  {\jdeqtp\Gamma{\ind{A+B}(z.C,x.c,y.d,\inr(b))}{d[b/y]}{C[\inr(b)/z]}}
\end{mathparpagebreakable}
%
In $\ind{A+B}$, $z$ is bound in $C$, $x$ is bound in $c$, and $y$ is bound in
$d$.

\subsection{The empty type \texorpdfstring{$\emptyt$}{0}}

\index{type!empty|(}%

\begin{mathparpagebreakable}
  \inferrule*[right=$\emptyt$-\rform]
  {\wfctx\Gamma}
  {\oftp\Gamma\emptyt{\UU_i}}
\and
  \inferrule*[right=$\emptyt$-\relim]
  {\oftp{\Gamma,\tmtp x\emptyt}{C}{\UU_i} \\ \oftp\Gamma{a}{\emptyt}}
  {\oftp\Gamma{\ind{\emptyt}(x.C,a)}{C[a/x]}}
\end{mathparpagebreakable}
%
In $\ind{\emptyt}$, $x$ is bound in $C$. The empty type has no introduction rule and no computation rule.

\index{type!empty|)}%

\subsection{The unit type \texorpdfstring{$\unit$}{1}}
\label{sec:more-formal-unit}

\index{type!unit|(}%

\begin{mathparpagebreakable}
  \inferrule*[right=$\unit$-\rform]
  {\wfctx\Gamma}
  {\oftp\Gamma\unit{\UU_i}}
\and
  \inferrule*[right=$\unit$-\rintro]
  {\wfctx\Gamma}
  {\oftp\Gamma{\ttt}{\unit}}
\and
  \inferrule*[right=$\unit$-\relim]
  {\oftp{\Gamma,\tmtp x\unit}{C}{\UU_i} \\
   \oftp{\Gamma}{c}{C[\ttt/x]} \\
   \oftp\Gamma{a}{\unit}}
  {\oftp\Gamma{\ind{\unit}(x.C,c,a)}{C[a/x]}}
\and
  \inferrule*[right=$\unit$-\rcomp]
  {\oftp{\Gamma,\tmtp x\unit}{C}{\UU_i} \\
   \oftp{\Gamma}{c}{C[\ttt/x]}}
  {\jdeqtp\Gamma{\ind{\unit}(x.C,c,\ttt)}{c}{C[\ttt/x]}}
\end{mathparpagebreakable}
%
In $\ind{\unit}$ the variable $x$ is bound in $C$.

Notice that we do not postulate a judgmental uniqueness principle for the unit
type; see \cref{sec:finite-product-types} for a proof of the corresponding
propositional uniqueness statement.

\index{type!unit|)}%

\subsection{The natural number type}

\index{natural numbers|(}%

We give the rules for natural numbers, following \cref{sec:inductive-types}.

\begin{mathparpagebreakable}
  \def\premise{
     \oftp{\Gamma,\tmtp x{\N}}{C}{\UU_i} \\
     \oftp\Gamma{c_0}{C[0/x]} \\
     \oftp{\Gamma,\tmtp{x}\N,\tmtp y C}{c_s}{C[\suc(x)/x]}}
  %
  \inferrule*[right=$\N$-\rform]
  {\wfctx\Gamma}
  {\oftp\Gamma{\N}{\UU_i}}
\and
  \inferrule*[right=$\N$-\rintro${}_1$]
  {\wfctx\Gamma}
  {\oftp\Gamma{0}{\N}}
\and
  \inferrule*[right=$\N$-\rintro${}_2$]
  {\oftp\Gamma{n}{\N}}
  {\oftp\Gamma{\suc(n)}{\N}}
\and
  \inferrule*[right=$\N$-\relim]
  {\premise \\ \oftp\Gamma{n}{\N}}
  {\oftp\Gamma{\ind{\N}(x.C,c_0,x.y.c_s,n)}{C[n/x]}}
\and
  \inferrule*[right=$\N$-\rcomp${}_1$]
  {\premise}
  {\jdeqtp\Gamma{\ind{\N}(x.C,c_0,x.y.c_s,0)}{c_0}{C[0/x]}}
\and
  \inferrule*[right=$\N$-\rcomp${}_2$]
  {\premise \\ \oftp\Gamma{n}{\N}}
  {\Gamma\vdash
    {\begin{aligned}[t]
      &\ind{\N}(x.C,c_0,x.y.c_s,\suc(n)) \\
      &\quad \jdeq c_s[n,\ind{\N}(x.C,c_0,x.y.c_s,n)/x,y] : C[\suc(n)/x]
    \end{aligned}}}
\end{mathparpagebreakable}
%
In $\ind{\N}$, $x$ is bound in $C$, and $x$ and $y$ are bound in $c_s$.

Other inductively defined types follow the same general scheme.

\index{natural numbers|)}%

\subsection{Identity types}

\label{sec:more-formal-identity}

\index{type!identity|(}%

The presentation here corresponds to the (unbased) path induction principle for identity types in
\cref{sec:identity-types}.

\begin{mathparpagebreakable}
  \inferrule*[right=$\idsym$-\rform]
  {\oftp\Gamma{A}{\UU_i} \\ \oftp\Gamma{a}{A} \\ \oftp\Gamma{b}{A}}
  {\oftp\Gamma{\id[A]{a}{b}}{\UU_i}}
\and
  \inferrule*[right=$\idsym$-\rintro]
  {\oftp\Gamma{A}{\UU_i} \\ \oftp\Gamma{a}{A}}
  {\oftp\Gamma{\refl a}{\id[A]aa}}
\and
  \inferrule*[right=$\idsym$-\relim]
  {\oftp{\Gamma,\tmtp xA,\tmtp yA,\tmtp p{\id[A]xy}}{C}{\UU_i} \\
   \oftp{\Gamma,\tmtp zA}{c}{C[z,z,\refl z/x,y,p]} \\
   \oftp\Gamma{a}{A} \\ \oftp\Gamma{b}{A} \\ \oftp\Gamma{p'}{\id[A]ab}}
  {\oftp\Gamma{\indid{A}(x.y.p.C,z.c,a,b,p')}{C[a,b,p'/x,y,p]}}
\and
  \inferrule*[right=$\idsym$-\rcomp]
  {\oftp{\Gamma,\tmtp xA,\tmtp yA,\tmtp p{\id[A]xy}}{C}{\UU_i} \\
   \oftp{\Gamma,\tmtp zA}{c}{C[z,z,\refl z/x,y,p]} \\
   \oftp\Gamma{a}{A}}
  {\jdeqtp\Gamma{\indid{A}(x.y.p.C,z.c,a,a,\refl a)}{c[a/z]}{C[a,a,\refl a/x,y,p]}}
\end{mathparpagebreakable}
%
In $\indid{A}$, $x$, $y$, and $p$ are bound in $C$, and $z$ is bound in
$c$.

\index{type!identity|)}%

\subsection{Definitions}

\index{definition}%

Although the rules we listed so far allows us to construct everything we need directly, we
would still like to be able to use named constants, such as $\isequiv$, as a matter of
convenience. Informally, we can think of these constants simply as
abbreviations, but the situation is a bit subtler in the formalization.

For example, consider function composition, which takes $f:A\to B$ and
$g:B\to C$ to $g\circ f:A\to C$. Somewhat unexpectedly, to make this work formally, $\circ$ must take as arguments not only $f$ and $g$, but also their types $A$, $B$, $C$:
%
\begin{narrowmultline*}
  {\circ} \defeq \lam{A:\UU_i}{B:\UU_i}{C:\UU_i}
  \narrowbreak
  \lam{g:B\to C}{f:A\to B}{x:A} g(f(x)).
\end{narrowmultline*}
%
From a practical perspective, we do not want to annotate each application of
$\circ$ with $A$, $B$ and $C$, as they are usually quite easily guessed from surrounding information. We would like to simply write $g\circ f$.
Then, strictly speaking, $g \circ f$ is not an abbreviation for $\lam{x : A} g(f(x))$,
because it involves additional \define{implicit arguments} which we want to suppress.
\index{implicit argument}

Inference of implicit arguments, typical ambiguity\index{typical ambiguity} (\cref{sec:universes}),
ensuring that symbols are only defined once, etc., are collectively called
\define{elaboration}. \index{elaboration, in type theory}
Elaboration must take place prior to checking a derivation, and is
thus not usually presented as part of the core type theory. However, it is
essentially impossible to use any implementation of type theory which does not
perform elaboration; see \cite{Coq,norell2007towards} for further discussion.

\section{Homotopy type theory}
\label{sec:hott-features}

In this section we state the additional axioms of homotopy type theory which distinguish it from standard Martin-L\"{o}f type theory: function extensionality, the
univalence axiom, and higher inductive types. We state them in the style
of the second presentation \cref{sec:syntax-more-formally}, although the first presentation \cref{sec:syntax-informally} could be used just as well.

\subsection{Function extensionality and univalence}

There are two basic ways of introducing axioms which do not introduce new syntax or judgmental equalities (function extensionality and univalence are of this form):
either add a primitive constant to inhabit the axiom, or prove all theorems which depend on the axiom by hypothesizing a variable that inhabits the axiom, cf.\ \cref{sec:axioms}.
While these are essentially equivalent, we opt for the former approach because we feel that the axioms of homotopy type theory are an essential part of the core theory.

\index{function extensionality}%
\cref{axiom:funext} is formalized by introduction of a constant $\funext$ which
asserts that $\happly$ is an equivalence:
%
\begin{mathparpagebreakable}
  \inferrule*[right=$\Pi$-\textsc{ext}]
  {\oftp\Gamma{f}{\tprd{x:A} B} \\
   \oftp\Gamma{g}{\tprd{x:A} B}}
  {\oftp\Gamma{\funext(f,g)}{\isequiv(\happly_{f,g})}}
\end{mathparpagebreakable}
%
The definitions of $\happly$ and $\isequiv$ can be found in~\eqref{eq:happly} and
\cref{sec:concluding-remarks}, respectively.

\index{univalence axiom}%
\cref{axiom:univalence} is formalized in a similar fashion, too:
%
\begin{mathparpagebreakable}
  \inferrule*[right=$\UU_i$-\textsc{univ}]
  {\oftp\Gamma{A}{\UU_i} \\
   \oftp\Gamma{B}{\UU_i}}
  {\oftp\Gamma{\univalence(A,B)}{\isequiv(\idtoeqv_{A,B})}}
\end{mathparpagebreakable}
%
The definition of $\idtoeqv$ can be found in~\eqref{eq:uidtoeqv}.

\subsection{The circle}

\index{type!circle}%

Here we give an example of a basic higher inductive type; others follow the same
general scheme, albeit with elaborations.

Note that the rules below do not precisely follow the pattern of the ordinary
inductive types in \cref{sec:syntax-more-formally}: the rules refer to the
notions of transport and functoriality of maps (\cref{sec:functors}), and the
second computation rule is a propositional, not judgmental, equality. These
differences are discussed in \cref{sec:dependent-paths}.

\begin{mathparpagebreakable}
  \inferrule*[right=$\Sn^1$-\rform]
  {\wfctx\Gamma}
  {\oftp\Gamma{\Sn^1}{\UU_i}}
\and
  \inferrule*[right=$\Sn^1$-\rintro${}_1$]
  {\wfctx\Gamma}
  {\oftp\Gamma{\base}{\Sn^1}}
\and
  \inferrule*[right=$\Sn^1$-\rintro${}_2$]
  {\wfctx\Gamma}
  {\oftp\Gamma{\lloop}{\id[\Sn^1]{\base}{\base}}}
\and
  \inferrule*[right=$\Sn^1$-\relim]
  {\oftp{\Gamma,\tmtp x{\Sn^1}}{C}{\UU_i} \\
   \oftp{\Gamma}{b}{C[\base/x]} \\
   \oftp{\Gamma}{\ell}{\dpath C \lloop b b} \\
   \oftp\Gamma{p}{\Sn^1}}
  {\oftp\Gamma{\ind{\Sn^1}(x.C,b,\ell,p)}{C[p/x]}}
\and
  \inferrule*[right=$\Sn^1$-\rcomp${}_1$]
  {\oftp{\Gamma,\tmtp x{\Sn^1}}{C}{\UU_i} \\
   \oftp{\Gamma}{b}{C[\base/x]} \\
   \oftp{\Gamma}{\ell}{\dpath C \lloop b b}}
  {\jdeqtp\Gamma{\ind{\Sn^1}(x.C,b,\ell,\base)}{b}{C[\base/x]}}
\and
  \inferrule*[right=$\Sn^1$-\rcomp${}_2$]
  {\oftp{\Gamma,\tmtp x{\Sn^1}}{C}{\UU_i} \\
   \oftp{\Gamma}{b}{C[\base/x]} \\
   \oftp{\Gamma}{\ell}{\dpath C \lloop b b}}
  {\oftp\Gamma{\Sn^1\text{-}\mathsf{loopcomp}}
    {\id {\apd{(\lamu{y:\Sn^1} \ind{\Sn^1}(x.C,b,\ell,y))}{\lloop}} {\ell}}}
\end{mathparpagebreakable}
%
In $\ind{\Sn^1}$, $x$ is bound in $C$. The notation ${\dpath C \lloop b b}$ for dependent paths was introduced in \cref{sec:dependent-paths}.
\index{rules of type theory|)}%

\section{Basic metatheory}
\index{metatheory|(}%

This section discusses the meta-theoretic properties of the type theory presented in 
\cref{sec:syntax-informally}, and similar results hold for \cref{sec:syntax-more-formally}. Figuring out which of these still hold when we add the features from \cref{sec:hott-features} quickly leads to open questions,\index{open!problem} as discussed at the end of this section.

Recall that \cref{sec:syntax-informally} defines the terms of type theory as
an extension of the untyped $\lambda$-calculus. The $\lambda$-calculus 
has its own notion of computation, namely the computation rule\index{computation rule!for function types}: 
\[
  (\lam{x} t)(u) \defeq t[u/x].
\]
This rule, together with the defining equations for the defined constants form
\emph{rewriting rules}\index{rewriting rule}\index{rule!rewriting} that determine reduction steps for a rewriting 
system. These steps yield a notion of computation in the sense that each rule
has a natural direction: one simplifies $(\lam{x} t)(u)$ by evaluating the
function at its argument.

Moreover, this system is \emph{confluent}\index{confluence}, that is, if $a$ simplifies in some
number of steps to both $a'$ and $a''$, there is some $b$ to which both $a'$ and
$a''$ eventually simplify. Thus we can define $t\conv u$ to mean that $t$ and
$u$ simplify to the same term.

(The situation is similar in \cref{sec:syntax-more-formally}: Although there
we presented the computation rules as undirected equalities $\jdeq$, we can give
an operational semantics by saying that the application of an eliminator to an
introductory form simplifies to its equal, not the other way around.)

Using standard techniques from type theory, it is possible to show that the system in \cref{sec:syntax-informally}
has the following properties:

\begin{thm}\label{thm:conversion-preserves-typing}
If $A : \UU$ and $A \conv A'$ then $A' : \UU$.
If $t:A$ and $t \conv t'$ then $t':A$.
\end{thm}

We say that a term is \define{normalizable}
\indexdef{term!normalizable}%
\index{normalization}%
\indexdef{normalizable term}%
(respectively, \define{strongly
normalizable})
\indexdef{term!strongly normalizable}%
\index{normalization!strong}%
\index{strong!normalization}%
if some (respectively, every), sequence of rewriting steps from the term
terminates.

\begin{thm}\label{thm:strong-normalization}
If $A : \UU$ then $A$ is strongly normalizable.
If $t:A$ then $A$ and $t$ are strongly normalizable.
\end{thm}

We say that a term is in \define{normal form}
\index{normal form}%
\index{term!normal form of}%
if it cannot be further
simplified, and that a term is \define{closed}
\index{closed!term}%
\index{term!closed}%
if no variable occurs freely in
it. A closed normal type has to be a primitive type, i.e., of the form
$c(\vec{v})$ for some primitive constant $c$ (where the list $\vec{v}$ of closed
normal terms may be omitted if empty, for instance, as with $\N$). In fact, we
can explicitly describe all normal forms:

\begin{lem}\label{lem:normal-forms}
  The terms in normal form can be described by the following syntax:
  % 
  \begin{align*}
    v & \production  k \mid \lam{x} v \mid c(\vec{v}) \mid f(\vec{v}), \\
    k &\production x \mid k(v) \mid f(\vec{v})(k),
  \end{align*}
  % 
  where $f(\vec{v})$ represents a partial application of the defined function $f$.
  In particular, a type in normal form is of the form $k$ or $c(\vec{v})$.
\end{lem}

\begin{thm}
  If $A$ is in normal form then the 
  judgment $A : \UU$ is decidable. If $A : \UU$ and $t$ is in normal form then the judgment
  $t:A$ is decidable.
\end{thm}

Logical consistency\index{consistency} (of the system in \cref{sec:syntax-informally}) follows
immediately: if we had $a:\emptyt$ in the empty context, then by
\cref{thm:conversion-preserves-typing,thm:strong-normalization}, $a$
simplifies to a normal term $a':\emptyt$. But by
\cref{lem:normal-forms} no such term exists.

\begin{cor}
 The system in \cref{sec:syntax-informally} is logically consistent.
\end{cor}

Similarly, we have the \emph{canonicity}\indexdef{canonicity} property that if $a:\N$ in the empty
context, then $a$ simplifies to a normal term $\suc^k(0)$ for some numeral $k$.

\begin{cor}
 The system in \cref{sec:syntax-informally} has the canonicity property.
\end{cor}

Finally, if $a,A$ are in normal form, it is \emph{decidable} whether $a:A$; in
other words, because type-checking amounts to verifying the correctness of a
proof, this means we can always ``recognize a correct proof when we see one''.

\begin{cor}
The property of being a proof in the system in \cref{sec:syntax-informally} is decidable.
\end{cor}

\mentalpause

The above results do not apply to the extended system of homotopy type
theory (i.e., the above system extended by \cref{sec:hott-features}), since
occurrences of the univalence axiom and constructors of higher inductive types
never simplify, breaking \cref{lem:normal-forms}. It is an open question\index{open!problem}
whether one can simplify applications of these constants in order to restore
canonicity. We also do not have a schema describing all permissible higher
inductive types, nor are we certain how to correctly formulate their rules
(e.g., whether the computation rules on higher constructors should be judgmental
equalities).

The consistency\index{consistency} of Martin-L\"{o}f type theory extended with univalence and higher
inductive types could be shown by inventing an appropriate normalization procedure, but currently
the only proofs that these systems are consistent are via semantic models---for
univalence, a model in Kan\index{Kan complex} complexes due to Voevodsky \cite{klv:ssetmodel}, and
for higher inductive types, a model due to Lumsdaine and Shulman \cite{ls:hits}.

Other metatheoretic issues, and a summary of our current results, are discussed
in greater length in the ``Constructivity'' and ``Open problems'' sections of
the introduction to this book.

\index{metatheory|)}%

\sectionNotes\label{subsec:general-remarks}

% This presentation is strongly inspired by two  Martin-L\"of 1972 and 1973.

The system of rules with introduction (primitive constants) and elimination
and computation rules (defined constant) is inspired by Gentzen natural
deduction. The possibility of strengthening the elimination rule for
existential quantification was indicated in \cite{howard:pat}. The
strengthening of the axioms for disjunction appears in \cite{Martin-Lof-1972},
and for absurdity elimination and identity type in \cite{Martin-Lof-1973}. The
$W$-types were introduced in \cite{Martin-Lof-1979}. They generalize a notion
of trees introduced by \cite{Tait-1968}.
\index{Martin-L\"of}%

%inspired from unpublished work of Spector.

The generalized form of primitive recursion for natural numbers and ordinals
appear in \cite{Hilbert-1925}. This motivated G\"odel's system $T$,
\cite{Goedel-T-1958}, which was analyzed by \cite{Tait-1966}, who used,
following \cite{Goedel-T-1958}, the terminology ``definitional equality'' for
conversion: two terms are \emph{judgmentally equal} if they reduce to a
common term by means of a sequence of applications of the reduction
rules. This terminology was also used by de Bruijn \cite{deBruijn-1973} in his
presentation of \emph{AUTOMATH}.\index{AUTOMATH}

Our second presentation comprises fairly standard presentation of
intensional Martin-L\"{o}f type theory, with some additional features needed in
homotopy type theory. Compared to a reference presentation of
\cite{hofmann:syntax-and-semantics}, the type theory of this book has a few
non-critical differences:
%
\begin{itemize}
\item universes \`{a} la Russell, in the sense of
\cite{martin-lof:bibliopolis}; and
\item judgmental $\eta$ and function extensionality for $\Pi$ types;
\end{itemize}
and a few features essential for homotopy type theory:
\begin{itemize}
\item the univalence axiom; and
\item higher inductive types.
\end{itemize}
%
As a matter of convenience, the book primarily defines functions by induction
using definition by \emph{pattern matching}.
\index{pattern matching}%
\index{definition!by pattern matching}%
It is possible to formalize the
notion of pattern matching, as done in \cref{sec:syntax-informally}. However, the
standard type-theoretic presentation, adopted in \cref{sec:syntax-more-formally}, is to introduce a single \emph{dependent
eliminator} for each type former, from which functions out of that type must be
defined. This approach is easier to formalize both syntactically and
semantically, as it amounts to the universal property of the type former.
The two approaches are equivalent; see \cref{sec:pattern-matching} for a
longer discussion.

\index{type theory!formal|)}%
\index{formal!type theory|)}%


%%% Local Variables: 
%%% mode: latex
%%% TeX-master: "hott-online"
%%% End: 
